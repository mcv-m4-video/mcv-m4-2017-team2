\section{Related work}
\label{sec:related}

In literature, many approaches have been studied to address the problem of vehicle tracking and speed estimation. On the one hand, we find multimodal approaches. Just to cite a few works published this year, in \cite{tian2017reliable} and \cite{tian2017self} the authors use two categories of sensors including IR Lidar and IR camera, and they fuse three detection techniques ---Time of Flight (ToF) based, vision based and Laser spot flow based--- in a Kalman filter framework. In \cite{wei2017adaptable} the authors propose a system based on tri-axial anisotropic magnetoresistive sensors and wireless sensor network, and in \cite{jinturkar2016real} the authors use three AMR sensors that give height of vehicle and speed estimation and number of vehicle passing near range of sensors scheme.\\

\noindent On the other hand, many approaches are based solely on video cameras which  benefit greatly from the recent advances in automated video analysis. Just to name a few, in \cite{zaki2017demonstrating}, the authors perform region based detection and featured based tracking. Similarly, in \cite{yabo2016vehicle} the authors detect moving vehicles through background/foreground segmentation techniques and estimate vehicles speed per class using feature tracking and nearest neighbors algorithms. In \cite{khilar2016novel}, vehicle detection is based on motion detection, using kernel density estimation in a pixel-based technique. Furthermore, the authors use a 3D pose estimation to handle occlusions, and a Kalman filter to perform the tracking. \cite{moremoving} presents a pyramidal approach of Lucas-Kanade algorithm to estimate motion vectors, and tracking of moving objects is performed using Kalman filter, tracking single or multiple moving objects. 

